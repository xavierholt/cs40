\documentclass{article}
\usepackage[utf8]{inputenc}
\usepackage{amssymb}
\usepackage{enumitem}
\usepackage{multicol}
\usepackage{tabularx}

\title{CS 40 Homework 3}
% Yeah, this is a hack...
\author{Due Tuesday, February 14th}
\date{Winter 2023}

\begin{document}

\maketitle

\noindent
A note on proofs:
Make sure you state what you're proving, what proof method you're using, and---once your proof is complete---what your conclusion is.
If you use a lot of symbols, a summary of your approach may be helpful to the reader.

\section{Inclusion}

\begin{multicols}{3}
\begin{itemize}
    % \setlength{\itemsep}{0pt}
    \item $A = \emptyset$
    \item $B = \{0\}$
    \item $C = \{\emptyset\}$
    \item $D = \{0, \emptyset\}$
    \item $E = \{\{0\}\}$
    \item $F = \{\{\emptyset\}\}$
    \item $G = \wp(\emptyset)$
    \item $H = \wp(\{\emptyset\})$
    \item $I = \wp(\{0, \emptyset\})$
\end{itemize}
\end{multicols}

\noindent
Let $X$ be a set containing the nine sets defined above.

\begin{enumerate}[label=\textbf{\alph*.}]
    % \setlength{\itemsep}{0pt}
    \item For each set $S$ in $\{G, H, I\}$, list every set $T$ in $X$ for which $T \in S$.
    \item For each set $S$ in $\{G, H, I\}$, list every set $T$ in $X$ for which $T \subset S$.
\end{enumerate}


\section{Counting}

Let $P = \{2, 3, 5, 7, 11, ..., 89, 97\}$ be a set containing the 25 prime numbers less than 100.
Write an expression that evaluates to the number of elements in $P^3 \times \wp(\wp(P \times P))$.
Explain your answer.


\section{Powersets}

Prove or disprove the following statements about arbitrary sets $A$ and $B$.

\begin{enumerate}[label=\textbf{\alph*.}]
    % \setlength{\itemsep}{0pt}
    \item $\forall A, B : \wp(A) \cap \wp(B) = \wp(A \cap B)$
    \item $\forall A, B : \wp(A) \cup \wp(B) = \wp(A \cup B)$
\end{enumerate}

\pagebreak
\section{Tuples}

Tuples can be used as ``record'' datatypes.
These are collections of related information, like structs or classes in a program, or rows in a database table.

Let $C$ be a set of cities and $P$ be a set of people.
Give an example of data that would be stored as each of the following types of tuples.

\begin{multicols}{3}
\begin{enumerate}[label=\textbf{\alph*.}]
    % \setlength{\itemsep}{0pt}
    \item $P \times \mathbb{Z}$
    \item $C \times \mathbb{R}$
    \item $P \times P$
    \item $P \times C \times \mathbb{N}$
    \item $C \times C \times \mathbb{N}$
    \item $C \times \mathbb{R} \times \mathbb{R}$
\end{enumerate}
\end{multicols}

\noindent
Make sure you use the correct types of integers.
Don't use $\mathbb{Z}$ to store age, for example, because that would imply that age might be negative; age should be stored as a natural number.
You can ignore the distinction between $\mathbb{Q}$ and $\mathbb{R}$


\section{Relations}

You can think of the binary operators from propositional logic as relations on the set of Booleans.
In this interpretation, two truth values are ``related'' by an operator if and only if the result of that operation is true.
\vspace{1em}

\noindent
\textbf{a.}
Write the following operations as binary relations.
Use truth set form, with each relation represented by the subset of $\mathbb{B} \times \mathbb{B}$ where the relation holds.
\vspace{1em}

\begin{centering}
\begin{tabularx}{3in}{XXXXX}
    $\wedge$ & $\vee$ & $\oplus$ & $\rightarrow$ & $\leftrightarrow$ \\
\end{tabularx}\par
\end{centering}
\vspace{1em}

\noindent
\textbf{b.}
Then determine which of the following properties apply to each operation.
You don't need to justify your answers.

\begin{multicols}{3}
\begin{itemize}
    % \setlength{\itemsep}{0pt}
    \item Reflexive
    \item Irreflexive
    \item Symmetric
    \item Antisymmetric
    \item Asymmetric
    \item Transitive
\end{itemize}
\end{multicols}


\section{Equivalence}

Consider approximate equality $(a \approx b)$ as a binary relation on the real numbers.
You might expect this to be an equivalence relation---but for this relation to be useful, it cannot be an equivalence relation.
Explain why not.


\section{Cycles}

Let $\succ$ be a binary relation.
We say that $\succ$ contains a cycle if there is a series of elements such that $a \succ b \succ c \succ ... \succ z \succ a$.
Prove that if $\succ$ is a strict partial order, it cannot contain a cycle.

\end{document}
