\documentclass{article}
\usepackage[utf8]{inputenc}
\usepackage{enumitem}

\title{CS 40 Homework 1}
% Yeah, this is a hack...
\author{Due Tuesday, January 24th}
\date{Winter 2023}

\begin{document}

\maketitle

\section{Encode}

Write the following sentences as logical expressions.
Create a variable for each atomic proposition, then use logical operators to connect your variables into longer expressions.
Your atomic propositions should be positive (A \textit{is} B; C \textit{will} D) rather than negative (A \textit{is not} B; C \textit{will not} D); each expression will use at least one logical operator.

\begin{enumerate}[label=\textbf{\alph*.}]
    \item Amy is not a banker.
    \item Bob speaks neither French nor Spanish.
    \item Carrie and Derek like each other.
    \item Erika will either buy a boat or a plane.
    \item Fred will come to dinner if George does, but not otherwise.
    \item Hailey might go to Macau, but only if Isabel will go too.
\end{enumerate}


\section{Decode}

Assume the following propositions.

\begin{itemize}
    \setlength{\itemsep}{0pt}
    \item $P = $ A particle was detected.
    \item $Q = $ The poison was not released.
    \item $R = $ The cat is dead.
\end{itemize}

\noindent
Rewrite the following expressions in English.

\begin{enumerate}[label=\textbf{\alph*.}]
    \item $\neg P$
    \item $R \oplus Q$
    \item $P \rightarrow \neg Q$
    \item $\neg R \rightarrow Q$
    \item $Q \leftrightarrow \neg P$
    \item $P \wedge \neg P$
\end{enumerate}


\section{Truth Tables}

Construct truth tables for the following expressions.
Add one row (or column) to the table for every operator as it is introduced.
Label each expression as a tautology, contradiction, or contingency.

\begin{enumerate}[label=\textbf{\alph*.}]
    \item $(p \rightarrow q) \leftrightarrow (\neg q \wedge p)$
    \item $(p \rightarrow q) \leftrightarrow (\neg q \rightarrow \neg p)$
    \item $(p \rightarrow q) \rightarrow (q \rightarrow p)$
\end{enumerate}

\noindent
Construct truth tables for the following deductions.
Add one row (or column) for every operator as it is introduced.
Label each deduction as valid or invalid.

\begin{enumerate}[label=\textbf{\alph*.}]
    \setcounter{enumi}{3}
    \item $P \vee Q, \neg Q \vdash P$
    \item $P, P \rightarrow Q \vdash Q$
    \item $\neg P, P \rightarrow Q \vdash \neg Q$
\end{enumerate}


\section{Operators}

The logical operators that you're familiar with are convenient because they correspond nicely to natural language.
However, not all of them are necessary.
In fact, you can build an equivalent logic system with only a single operator.
Consider the operator $a \downarrow b \equiv \neg(a \vee b)$.
Show that the following expressions can be rewritten using only the $\downarrow$ operator.

\begin{enumerate}[label=\textbf{\alph*.}]
    \setlength{\itemsep}{0pt}
    \item $\neg R$
    \item $L \wedge R$
    \item $L \vee R$
    \item $L \oplus R$
    \item $L \rightarrow R$
    \item $L \leftrightarrow R$
\end{enumerate}


\section{Limitations}

There are some situations that can't be properly modeled in the logic we've discussed in class.
Assuming that the following premises are all true, use them to construct a sound argument---then argue that that argument reaches an unrealistic conclusion.

\begin{itemize}
    \setlength{\itemsep}{0pt}
    \item If I have at least five dollars, I can buy cake.
    \item If I have at least five dollars, I can buy beer.
    \item If I have exactly five dollars, I have at least five dollars.
    \item I have exactly five dollars.
\end{itemize}

\end{document}
