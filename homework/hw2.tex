\documentclass{article}
\usepackage[utf8]{inputenc}
\usepackage{amssymb}
\usepackage{enumitem}

\title{CS 40 Homework 2}
% Yeah, this is a hack...
\author{Due Monday, January 30th}
\date{Winter 2023}

\begin{document}

\maketitle

\section{Decode}

The following are sets.

\begin{itemize}
    \setlength{\itemsep}{0pt}
    \item $E = $ Edible things.
    \item $L = $ Things that Lisa likes.
    \item $M = $ Meats.
    \item $P = $ Poisons.
    \item $Y = $ Yogurts.
\end{itemize}

\noindent
Rewrite the following expressions in English.

\begin{enumerate}[label=\textbf{\alph*.}]
    \setlength{\itemsep}{0pt}
    \item $L = \emptyset$
    \item $M \subseteq E$
    \item $L = P \cap Y$
    \item $Y \setminus E \neq \emptyset$
    \item $L = E \cap \overline{M \cup P}$
    \item $M \bigtriangleup Y = M \cup Y$
\end{enumerate}


\section{Encode}

Write the following sentences in set notation.
Define variables for any sets you need in advance (do this once; use these variables for the entire question).
Then rewrite the following sentences using only your variables, set operators, and equalities/inequalities.
Do not use any logical operators.

\begin{enumerate}[label=\textbf{\alph*.}]
    \setlength{\itemsep}{0pt}
    \item Some living things can fly.
    \item Not all flying things are alive.
    \item All crawling things are living things.
    \item Airplanes can fly, but they are not alive.
    \item Bees are living things that can fly and crawl.
    \item Dirigibles are not living things, and cannot crawl.
\end{enumerate}


\section{Proofreading}

The following proofs are unsound.
Explain why.

\begin{enumerate}[label=\textbf{\alph*.}]
    \item My parrot is not a mammal.
    \begin{enumerate}[label={(\arabic*})]
        \setlength{\itemsep}{0pt}
        \item Everything with four legs is a mammal.
        \item My parrot has two legs.
        \item Therefore, my parrot is not a mammal.
    \end{enumerate}

    \item Today is Saturday.

    I have a neighbor who plays tennis a lot.
    His name's Bob.
    Every Saturday, as long as it's sunny, he goes to our local park and plays tennis.
    Earlier this morning, I didn't know what day it was, but I knew it was the weekend, and it was sunny, so I went for a walk.
    Then I saw Bob playing tennis in the park, so I knew it must be Saturday.

    \item All even numbers are prime.
    \begin{enumerate}[label={(\arabic*})]
        \setlength{\itemsep}{0pt}
        \item Let $E(x)$ be the proposition ``$x$ is even.''
        \item Let $P(x)$ be the proposition ``$x$ is prime.''
        \item \label{taut} We know that $(A \rightarrow B) \vee (B \rightarrow A)$; this is a tautology.
        \item \label{subd} Substituting into \ref{taut}, we get $(E(x) \rightarrow P(x)) \vee (P(x) \rightarrow E(x))$.
        \item \label{cont} The number 13 is prime but not even, so $P(x) \rightarrow E(x)$ must be false.
        \item By disjunctive syllogism from \ref{subd} and \ref{cont}, we can conclude that $E(x) \rightarrow P(x)$ must be true.
        \item Therefore all even numbers are prime.
    \end{enumerate}
\end{enumerate}


\section{Addition}

Prove that for every positive irrational number $x$, there is at least one other positive irrational number $y$ such that $x$ and $y$ sum to an integer.


\section{Division}

Prove the following statement.

$$
\forall y \in \mathbb{R} \setminus \mathbb{Q} \left(
    \forall x \in \mathbb{Q} \left(
        x \neq 0
        \rightarrow
        \frac{x}{y} \in \mathbb{R \setminus Q}
    \right)
\right)
$$


\section{Multiplication}

Let $\Pi(A)$ be the product of the elements in set $A$.
Prove the following statement.

$$
\forall S \left(
    S \subseteq \mathbb{Z} \wedge |S| \geq 3
    \rightarrow
    \exists S' \subseteq S \left(
        S' \neq \emptyset \wedge \Pi(S') \in \mathbb{Z}^+
    \right)
\right)
$$

\end{document}
