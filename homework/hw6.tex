\documentclass{article}
\usepackage[utf8]{inputenc}
\usepackage{amsmath}
\usepackage{amssymb}
\usepackage{enumitem}
\usepackage[hidelinks]{hyperref}
\usepackage{multicol}
\usepackage{tabularx}

% https://tex.stackexchange.com/a/286470
\font\domino=domino
\def\die#1{{\domino#1}}

\title{CS 40 Homework 6}
% Yeah, this is a hack...
\author{Due Wednesday, March 15th}
\date{Winter 2023}

\begin{document}

\maketitle

% \noindent
% A note on proofs:
% Make sure you state what you're proving, what proof method you're using, and---once your proof is complete---what your conclusion is.
% If you use a lot of symbols, a summary of your approach may be helpful to the reader.


\section{Dice I}

You have some standard six-sided dice.
What is the probability of rolling a total of ten when you roll...

\begin{enumerate}[label=\textbf{\alph*.}]
    % \setlength{\itemsep}{0pt}
    \item ...one die?
    \item ...two dice?
    \item ...three dice?
\end{enumerate}


\section{Dice II}

Alice and Bob like to play a very simple dice game:
Each player gets one die; they roll their dice, and whoever rolls the higher number wins.
If it's a tie, they re-roll until someone wins.

Normally they play at Alice's house.
Alice has standard dice, so the game is fair: both players are equally likely to win.
But today they're playing at Bob's house, and Bob's dice are unusual.
There are three dice to choose from:

\begin{itemize}
    \item Die X: \die1, \die1, \die3, \die5, \die5, \die6
    \item Die Y: \die2, \die3, \die3, \die4, \die4, \die5
    \item Die Z: \die1, \die2, \die2, \die4, \die6, \die6
\end{itemize}
\vspace{4pt}

\noindent
\textbf{a.}
What is the expected value of each die?
\vspace{1em}

\noindent
\textbf{b.}
Consider each possible pair of dice.
What is the probability of each die winning, tying, or losing a roll?
\vspace{1em}

\noindent
\textbf{c.}
How can Alice give herself an advantage?


\section{Coins I}

You have an unbalanced coin:
When you flip the coin, getting heads (55\%) is slightly more likely than getting tails (45\%).
You flip this coin seven times.

\begin{enumerate}[label=\textbf{\alph*.}]
    % \setlength{\itemsep}{0pt}
    \item What is the probability you get seven heads?
    \item What is the probability you get at least one heads?
    \item What is the probability you get exactly three heads?
    \item What is the probability you get more heads than tails?
\end{enumerate}

\section{Coins II}

Consider the following gambling game:
You pay a one-time fee, and then you flip a fair coin.
Every time you flip heads, you get one dollar and you get to flip again.
As soon as you flip tails, the game is over.

What fee would set the expected value of this game to zero?


\section{Doors}

A game show host shows you two doors, and asks you to pick one.
Both doors have money behind them, any you get to keep whatever you find behind the door you choose.
There's no way to tell what's behind each door, so you pick one at random.

Then the host tells you that one door has twice as much money behind it than the other.
He asks if you would like to switch doors.
You figure:

\begin{enumerate}[label=\textbf{\roman*.}]
    \item Your current door contains some unknown amount of money.
    You don't know how much, so you call it $n$ dollars.

    \item If you switch doors, you might double your money, but alternatively, you might loose half of it.

    \item Since you chose at random, both of these possibilities are equally likely.

    \item You calculate the expected value of switching, and get:
    $$\left(\frac{1}{2} \times 2n\right) + \left(\frac{1}{2} \times \frac{n}{2}\right) \, = \, n + \frac{n}{4} \, = \, 1.25n$$

    \item This is higher than your current value of $n$, so you should switch.
\end{enumerate}

\noindent
But then you realize that if you did switch doors, the host could ask you if you wanted to switch a second time---and you would use the exact same logic and decide to switch again.
In fact, you could end up switching forever.
This is absurd, so there must be something wrong with your logic.
What was the flaw?


\section{Werewolf}

Amy thinks her boyfriend might be a werewolf.
She knows that only 1 in 10,000 people in the general population are werewolves, but she wants to be sure.
So she buys a werewolf test kit---the good brand, with a 98\% true positive rate and a 97\% true negative rate---and tests him while he's sleeping.

\begin{enumerate}[label=\textbf{\alph*}]
    \item If the test comes back negative, what is the probability that he really is a werewolf?

    \item If the test comes back positive, what is the probability that he really is a werewolf?

    \item If three subsequent tests all come back positive, what's the probability that he's a werewolf?
\end{enumerate}

\end{document}
