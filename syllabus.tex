\documentclass{article}
\usepackage[utf8]{inputenc}
\usepackage[hidelinks]{hyperref}

\title{CS 40 Syllabus}
\author{Kevin Burk}
\date{Winter 2023}

\begin{document}

\maketitle

\section{Introduction}

Welcome to CS 40!
The official title of this class is Foundations of Computer Science, which is a bit vague.
I prefer to think of it as Discrete Mathematics.
The goal is to provide an overview of the mathematics that serve as the foundation for the theory of computation.

We'll be using two websites for everything in the class.
Piazza\footnote{\url{https://www.gradescope.com}} is the class discussion forum.
If you have any questions about the class or the assignments, you can ask them there.
You can also find the homework on Piazza, under the Resources tab.
You'll turn in your homework on Gradescope.\footnote{\url{https://piazza.com}}
This is also where you'll see your grades---for both homework and exams---and where you can make regrade requests.

I'll be using the textbook \textit{Discrete Mathematics and its Applications} by Kenneth Rosen as my personal reference material.
The textbook isn't required for the class, but it may be useful, as I don't typically post lecture notes.


\section{Topics}

This class will provide an overview of a variety of topics, the most important of which can be found below.
The exact ordering and emphasis may change based on time constraints and instructorial caprice.

\begin{itemize}
    \setlength\itemsep{0em}
    \item Logic, propositional and first-order.
    \item Sets, finite and otherwise.
    \item Proofs, including inductive proofs.
    \item Functions and relations.
    \item Groups, rings, and fields.
    \item Combinatorics, counting, and probability.
    \item Algorithms, and the analysis thereof.
    \item Computability and complexity.
\end{itemize}


\section{Assignments}

This is a mathematics class, with written assignments and exams.
Each week after the first will have one or the other, for a total of two midterms, one final, and approximately seven written homework assignments.
There are no programming assignments.

Homework assignments will be published on Piazza, and should be turned in online via Gradescope.
You may work on homework assignments in pairs; if you choose to do this, submit one file to Gradescope and add both partners' names to that submission.

The point of the assignments is not only to solve the problems given, but also to clearly communicate the correctness of your solution.
Answers that are unintelligible for any reason---including, but not limited to, poor handwriting and poor grammar---will not receive credit.
You are strongly encouraged to sidestep the handwriting problem entirely and typeset your assignments in \LaTeX{}, which is available on the CSIL computers, as well as online at Overleaf.\footnote{\url{https://www.overleaf.com}}

You may submit homework up to three days late, but at a penalty of 10\% per day.
If an assignment is due on Monday, a submission on Tuesday will receive at most a 90\%, Wednesday an 80\%, and Thursday a 70\%; submissions on Friday or later will not be accepted.

Grades should be released within a week of the assignment's due date or the date of the test.
Regrade requests can be made on Gradescope, and must be made within one week of grades being released.


\section{Grading}

Your final score will be made up of four weighted components:

\begin{itemize}
    \setlength\itemsep{0em}
    \item 25\% Homework
    \item 20\% Midterm One
    \item 25\% Midterm Two
    \item 30\% Final Exam
\end{itemize}

I may decide to curve.
If I do, it will be based on the final scores, and it will always be in your favor.
Getting 90\% of the possible points guarantees you at least an A-, 80\% guarantees you at least a B-, and so on.

If you decide to cheat, however, your grade will go down.
Cheating on any of the homework assignments will result in a grade of zero on that assignment, and your final grade in the class will be reduced by one letter.
If you do it a second time---or if you cheat on any of the exams---you will fail the class.

\end{document}
